%\documentclass[twocolumns]{article}
\documentclass{article}%especifica o tipo de documento que tenciona escrever: carta, artigo, relatório... neste caso é um artigo

\usepackage[portuges]{babel}%Babel -- irá activar automaticamente as regras apropriadas de hifenização para a língua todo o
                                   %-- o texto gerado é automaticamente traduzido para Português.
                                   %  Por exemplo, “chapter” irá passar a “capítulo”, “table of contents” a “conteúdo”.
                                   %portuges -- específica para o Português.
\usepackage[utf8]{inputenc} % define o encoding usado texto fonte (input)

\usepackage{graphicx} %permite incluir graficos, tabelas, figuras

\parindent=0pt
\parskip=2pt


\title{Um Exemplo de Artigo em \LaTeX} %Titulo do documento
\author{Cristina Araújo %define o nome do 1. autor
        \\ Universidade do Minhho %...e sua filiação
        \and            %obrigatorio para separar autores
        Sandra Lopes\thanks{Bolseiro da FCT} % coloca em roda pe a informação marcada
        \\Escola FFF\\ Fafe
        \and
        Pedro Henriques\\Escola de Engenharia\\ Braga
 } %autores do documento
\date{ (\today)} %data



\begin{document} % corpo do documento

\maketitle %apresentar titulo, autor e data

\begin{abstract} % resumo do documento
\noindent Isto é um resumo do artigo. \\ %\noindent -- criar um parágrafo não indentado
O objectivo é sintetizar em 2 ou 3 parágrafos a ideia principal descrita no artigo.
\end{abstract}

%\tableofcontents % Insere a tabela de indice

\section{Introdução} \label{sec:introducao} %etiqueta para fazer referência cruzada às secçoes
Enquadramento---Contexto do Problema e Motivação para o trabalho---do documento.
% FIM DE PARAGRAFO E LINHA EM BRANCO
Objetivos claros e concretos do artigo.
% FIM DE PARAGRAFO E LINHA EM BRANCO

Agora virá algum texto claro e apelativo à leitura.

Aqui vai um exemplo de matemática com formas INLINE
 $ \alpha\   \oint   \Psi \Delta \pi \  \varphi\ \phi\ \wedge \biguplus\ {\AE}  $

Aqui vai um exemplo de matemática  EM DESTAQUE
\[  \alpha\   \pi \  \varphi\ \phi\ \wedge \$   \]

O artigo está organizado da seguinte forma.
Na Secção~\ref{sec:background} %referência cruzada à secçao com a etiqueta identifacada
falo dos conceitos básicos necessários para perceber o texto.
Depois na Secção~\ref{sec:concbasicos} continuo a explicação com mais detalhe.

Na Secção~\ref{sec:proposta} apresentamos a proposta dos autores para resolver o problema
e na Secção~\ref{sec:implementa} discute-se a sua implementação e teste.

Por fim a Secção~\ref{sec:conclusao} apresenta uma síntese do artigo e as conclusões discutindo os resultados obtidos;
traça ainda linhas para trabalho futuro-
%~\ref{proposta}-- substituí \ref pelo número da secção, subsecção, figura, tabela ou teorema após o respectivo comando \label foi invocado.

Depois da organização começa mesmo o corpo do documento!

\section{Background} \label{sec:background}
Secção onde se apresentam os conceitos e definições básicas para entender o resto do texto.
Discute-se também o state-of.the-art e os trabalhos relacionados com o nosso.

\section{Conceitos básicos em XXX} \label{sec:concbasicos}
\subsection{Conceitos básicos em XXX.1} %subseccao
Aqui vai um exemplo de uma lista numerada
\begin{itemize}  % EXEMPLO DE LISTA DE ITEMS SIMPLES assinalados com PONTO
\item EXEMPLO DE ITEM SIMPLES
\item segunda característica desta fase
\item terceira característica desta fase
\end{itemize}

\subsection{Conceitos básicos em XXX.2}
Seguem-se algumas definições fundamentais para se perceberem as ideias
defendidas a seguir:
\begin{description} % EXEMPLO DE lISTA DE ITEMS TIPO ENTRADAS COM DESCRIÇÃO
\item[conceito 1] texto com a descrição do  conceito 1
\item[conceito 2] descrição do  conceito 2
\end{description}

\subsection{Conceitos básicos em YYY}
Veja a Figura \ref{fig:figuraA}. %substituí \ref pelo número da figura

\section{A Proposta} \label{sec:proposta}
Como se vê na Figura~\ref{fig:figuraA} bla bla bla

    \begin{figure} %insere figuras
     \begin{center} % insere a figura ao centro
         \includegraphics[width=0.5\textwidth]{persevering.png}
         % width=0.5\textwidth -- a figura é alterada de forma a que a largura seja 0.5 vezes a largura de um parágrafo normal (textwidth). A altura é calculada de forma a manter a relação altura/largura.
         \caption{Legenda da Figura} \label{fig:figuraA} %legenda da figura
        \end{center}
    \end{figure}

\section{A sua Implementação} \label{sec:implementa}
A Figura \ref{fig:figuraB}, que surge na pagina~\pageref{fig:figuraB}, % \pageref imprime o número da página onde o comando \label ocorreu
mostra o esquema seguido na implementação.

Os resultados finais desta implementação serão discutidos na Secção \ref{sec:conclusao}.

\begin{figure}[!htbp] %[especificação de colocação da figura]
                      % h -- aqui neste exacto local onde ocorreu no meio do texto.
                      % t -- no topo da página
                      % b -- no fundo (bottom) da página
                      % p -- numa página especial apenas com corpos flutuantes..
                      % ! sem considerar a maior parte dos parâmetros internos a que podem fazer com que o corpo flutuante não seja colocado
\begin{center}
    \includegraphics[width=\textwidth]{persevering.png}
    \caption{Legenda da Imagem} \label{fig:figuraB}
\end{center}
\end{figure}

\newpage

\section{Conclusão} \label{sec:conclusao}
Síntese do que foi dito.\\
Lista dos resultados atingidos:
\begin{enumerate} % EXEMPLO DE lISTA DE ITEMS ENUMERADOS
\item resultado um
      \begin{itemize}
        \item subresultado primeiro
        \item subresultado segundo
      \end{itemize}
\item resultado dois
\end{enumerate}
Conclusão final e Trabalho Futuro.

\end{document}


