\documentclass{article}%especifica o tipo de documento que tenciona escrever: carta, artigo, relatório... neste caso é um artigo

\usepackage[portuges]{babel}%Babel -- irá activar automaticamente as regras apropriadas de hifenização para a língua todo o
                                   %-- o texto gerado é automaticamente traduzido para Português.
                                   %  Por exemplo, “chapter” irá passar a “capítulo”, “table of contents” a “conteúdo”.
                                   %portuges -- específica para o Português e outras línguas com caracteres acentuados.
\usepackage[utf8]{inputenc} % input do encode  do teclado

\usepackage{graphicx} %permite incluir graficos, tabelas, figuras

%listing%
\usepackage{eurosym}
\usepackage{listings}
\usepackage{color}
%\usepackage{listingsutf8}

\definecolor{codegreen}{rgb}{0,0.6,0}
\definecolor{codegray}{rgb}{0.5,0.5,0.5}
\definecolor{codepurple}{rgb}{0.58,0,0.82}
\definecolor{backcolour}{rgb}{0.95,0.95,0.92}
\definecolor{lightgray}{rgb}{0.83,0.83,0.83}
\definecolor{dukeblue}{rgb}{0.0, 0.0, 0.61}
\definecolor{pakistangreen}{rgb}{0.0, 0.4, 0.0}
\definecolor{saddlebrown}{rgb}{0.55, 0.27, 0.07}

\lstdefinestyle{mystyle}{
    backgroundcolor=\color{backcolour},
    commentstyle=\color{saddlebrown},
    keywordstyle=\color{dukeblue},
    numberstyle=\tiny\color{codegray},
    stringstyle=\color{pakistangreen},
    basicstyle=\footnotesize,
    breakatwhitespace=false,
    breaklines=true,
    captionpos=b,
    keepspaces=true,
    numbers=left,
    numbersep=5pt,
    showspaces=false,
    showstringspaces=false,
    showtabs=false,
    inputencoding=utf8,
    extendedchars=true,
    escapeinside={\%*}{*)},
    literate={á}{{\'a}}1 {ã}{{\~a}}1 {â}{{\^a}}1  {é}{{\'e}}1 {ê}{{\^e}}1  {í}{{\'i}}1 {ó}{{\'o}}1 {õ}{{\~o}}1 {ô}{{\^o}}1 {ú}{{\'u}}1 {ç}{{\'ç}}1,
    tabsize=2
}

\lstset{style=mystyle}

\parindent=0pt
\parskip=2pt

\setlength{\oddsidemargin}{-1cm} %espaço entre o texto e a margem
\setlength{\textwidth}{18cm} %Comprimento do texto na pagina
\setlength{\headsep}{-1cm} %espaço entre o texto e o cabeçalho
\setlength{\textheight}{23cm} %altura do texto na pagina




\title{Notas sobre  Bibliografia em  Bib\TeX} %Titulo do documento
\author{Cristiana Araújo \and Pedro Rangel Henriques} %autores do documento
\date{ \today} %data



\begin{document} % corpo do documento

\maketitle %apresentar titulo, autor e data

Bib\TeX\ é uma ferramenta, complementar ao processador \LaTeX,  de formatação de referências bibliográficas
citadas em documentos \LaTeX.
Ela foi criada por Oren Patashnik e Leslie Lamport em 1985 para facilitar a separação da bibliografia
do texto que constitui o corpo central do documento,
seguindo o mesmo conceito de distinção entre o conteúdo e o estilo do texto utilizado no próprio \LaTeX.

Para se gerar com o Bib\TeX\ uma bibliografia que possa ser incluida num documento, é necessário criar
um outro ficheiro de texto à parte com o nome que se quiser mas terminado com a extensão \texttt{.bib}
(por exemplo, \texttt{arquivoBib.bib}) contendo uma base de dados bibliográfica, (textual) a qual poderá ser reutilizada
em qualquer outro documento \LaTeX\ que posteriormente se venha a criar.
De modo a que a base de dados com registos bibliográficos possa ser tratada pelo Bib\TeX\ é necessário
que cada registo seja escrito usando uma sintaxe própria e que contenha uma estrutura adequada, conforme se
vai explicar na próxima secção.

\section{Tipos de Referências em Bib\TeX} \label{sec:bibliografia} %referência cruzada para secçoes

Embora a sintaxe a usar para descrever cada referência bibliográfica seja sempre a mesma, os campos de cada registo
são variáveis e dependem do tipo de documento que se quer registar, existindo atualmente em Bib\TeX mais de quinze tipos distintos.

As entradas desta base de dados devem ter sempre a seguinte estrutura:

\begin{verbatim}
 @tipoEntrada{label,
    campo1 = {Valor do campo 1},
    campo2 = {Valor do campo 2},
    campo3 = {Valor do campo 3},
    [...]
    campo4 = {Valor do campo 4}
 }
\end{verbatim}

A seguir são listados os tipos de entrada mais relevantes (mais usados)  e os respetivos campos, mostrando-se ainda um exemplo para cada caso.
Convém notar que as entradas são compostas de campos obrigatórios (são aqueles sem os quais uma referência não
pode ser caracterizada) e
opcionais (são aqueles que não sendo essenciais ajudam a caracterizar mais adequadamente a referência).

\begin{description}
  \item[@article] Um artigo de um periódico ou revista.
  \begin{itemize}
    \item \emph{Campos obrigatórios:} autor, title, journal, year.
    \item \emph{Campos opcionais:} volume, number, pages, month, issn, publisher, doi, note.
  \end{itemize}

  \begin{lstlisting}
  @article{araujo:2018b,
    author    = {Cristiana Ara{\'u}jo and Pedro Rangel Henriques and Ricardo G. Martini},
    title     = {{Virtual Learning Spaces Creation Based on the Systematic Population of an Ontology}},
    journal   = {Journal of Information Systems Engineering \& Management (JISEM)},
    pages     = {1-11},
    volume    = {3},
    number    = {1},
    year      = {2018},
    month     = {Feb},
    issn      = {2468-4376},
    publisher = {Lectito and AISTI},
    doi       = {https://doi.org/10.20897/jisem.201807}
  }
   \end{lstlisting}


  \item[@book] Um livro inteiro publicado.
  \begin{itemize}
    \item \emph{Campos obrigatórios:} author ou editor; title; publisher; year.
    \item \emph{Campos opcionais:} volume ou number; series; address; edition; month; isbn; note.
  \end{itemize}

  \begin{lstlisting}
  @book{damas:1999,
    title     = {Linguagem C},
    author    = { Luís Damas},
    isbn      = {9789727221561},
    year      = {1999},
    publisher = {FCA}
  }
  \end{lstlisting}


  \item[@inBook]  Uma parte de um livro (capítulo dentro do livro).
  \begin{itemize}
    \item \emph{Campos obrigatórios:} author/editor, title, chapter/pages, publisher, year.
    \item \emph{Campos opcionais:} volume/number, series, type, address, edition, doi, month, note.
  \end{itemize}

  \begin{lstlisting}
  @inBook{Hopcroft:2006b,
    author  =  {John E. Hopcroft and Rajeev Motwani and Jeffrey Ullman},
    title   =  {Introduction to Automata Theory, Languages, and Computation},
    chapter =  {5 -- Context-Free Grammars and Languages},
    year    =  {2006},
    edition =  {3rd Ed.},
    publisher = {Addison-Wesley},
  }
  \end{lstlisting}


  \item[@misc]  Algo que não se encaixa em nenhum outro tipo (imagens, áudios, vídeos,...).
  \begin{itemize}
    \item \emph{Campos obrigatórios:} Nenhum.
    \item \emph{Campos opcionais:} author, title, howpublished, month, year, url, date, note.

  \end{itemize}

  \begin{lstlisting}
  @misc{knuth:1999,
    author    = {Donald Knuth},
    title     = {Knuth: Computers and Typesetting},
    year      = {1999}
  }
  \end{lstlisting}


\item[@online]  Utilizado para sites, páginas web...
\begin{itemize}
    \item \emph{Campos obrigatórios:} Nenhum.
    \item \emph{Campos opcionais:} author, title, howpublished, month, year, url, date, note.
  \end{itemize}

  \begin{lstlisting}
  @online{antlr:2016,
    author  = {ANTLR},
    title   = {{ANTLR}},
    howpublished = {\url{http://www.antlr.org/}},
    date    = {2016-09},
    year    = {2016},
    note    = {Accessed: 2016-09-14}
  }
  \end{lstlisting}


  \item[@inCollection]  Livro de uma série (colecção de livros, por exemplo, Uma Aventura).
  \begin{itemize}
    \item \emph{Campos obrigatórios:} author, title, booktitle, publisher, year.
    \item \emph{Campos opcionais:} editor, volume/number, series, type, chapter, pages, address, edition, month, note.
  \end{itemize}

  \begin{lstlisting}
  @inCollection{maskin:1985,
    author  = {Eric S. Maskin},
    title   = {The theory of implementation in {N}ash equilibrium: a survey},
    booktitle = {Social Goals and Social Organization},
    editor  = {Leonid Hurwicz and David Schmeidler and Hugo Sonnenschein},
    year    = {1985},
    pages   = {173-204},
    publisher = {Cambridge University Press},
    address = {Cambridge}
  }
  \end{lstlisting}


  \item[@proceedings] Livro de atas da conferência.
  \begin{itemize}
    \item \emph{Campos obrigatórios:} title, year.
    \item \emph{Campos opcionais:} editor, volume/number, series, address, publisher, month, note, organization.
  \end{itemize}

  \begin{lstlisting}
  @proceedings{deransart:1990,
    title  = {Attribute Grammars and their Applications},
    editor = {P. Deransart and M. Jourdan},
    organization = {INRIA},
    year   = {1990},
    month  = {Sep},
    publisher = {sv},
    note   = {Lecture Notes in Computer Science, nu. 461},
    annote = {compilacao, atributos}
  }
  \end{lstlisting}


  \item[@inProceedings]  Um artigo nas atas de conferência.
  \begin{itemize}
    \item \emph{Campos obrigatórios:} author, title, booktitle, year.
    \item \emph{Campos opcionais:} editor, volume/number, series, pages, address, month, isbn, organization, publisher, note.
  \end{itemize}

  \begin{lstlisting}
  @inProceedings{araujo:2017a,
    author   = {Cristiana Ara{\'u}jo  and Pedro Rangel Henriques and Ricardo G. Martini},
    title    = {{Automatizing Ontology Population to drive the navigation on Virtual Learning Spaces}},
    booktitle = {Atas da 12$^{a}$ Confer{\^e}ncia Ib{\'e}rica de Sistemas e Tecnologias de Informa\c{c}\~ao, CISTI'2017},
    editor   = {{\'A}lvaro Rocha and Br{\'a}ulio Alturas and Carlos Costa and Lu{\'i}s Paulo Reis and Manuel P{\'e}rez Cota},
    pages    = {781-786},
    year     = {2017},
    month    = {June},
    publisher = {AISTI--Associacao Ib{\'e}rica de Sistemas e Tecnologias de Informacao},
    isbn     = {978-989-98434-7-9},
    note     = {}
  }
  \end{lstlisting}


  \item[@mastersthesis] Uma tese de mestrado.
  \begin{itemize}
    \item \emph{Campos obrigatórios:} author, title, school, year.
    \item \emph{Campos opcionais:} type, address, month, note.
  \end{itemize}

  \begin{lstlisting}
  @mastersthesis{araujo:2016,
	author = {Cristiana Ara{\'u}jo},
	title  = {{Building the Museum of the Person Based on a combined CIDOC-CRM/ FOAF/ DBpedia Ontology}},
	year   = {2016},
    school = {Universidade do Minho},
    month  = {December},
    address= {Braga, Portugal},
    note   = {{MSc} dissertation}
  }
  \end{lstlisting}


  \item[@phdthesis]  Uma tese de doutoramento.
  \begin{itemize}
    \item \emph{Campos obrigatórios:} author, title, school, year.
    \item \emph{Campos opcionais:} type, address, month, note, annote.
  \end{itemize}

  \begin{lstlisting}
  @phdthesis{martini:2018,
    author = {Ricardo Giuliani Martini},
    title  = {{Formal Description and Automatic Generation of Learning Spaces based on Ontologies}},
    year   = {2018},
    month  = {Sep},
    school = {University of Minho},
    annote = {Ontologies, DSLs, Virtual Learning Spaces, Automatic Generation of Websites}
  }
  \end{lstlisting}


  \item[@techreport]  Relatório publicado por uma instituição.
  \begin{itemize}
    \item \emph{Campos obrigatórios:} author, title, institution, year.
    \item \emph{Campos opcionais:} type, number, address, month, note.
  \end{itemize}

  \begin{lstlisting}
  @techreport{araujo:2017,
	author    = {Cristiana Ara{\'u}jo},
	title     = {{Norma -- Simplex Project}},
    institution = {UNU-EGOV, United Nations University},
    year      = {2017},
    type      = {{Technical Report}},
	language  = {English}
  }
  \end{lstlisting}


  \item[@booklet] Um trabalho que é impresso mas não tem editora ou instituição patrocinadora (exemplo:folheto).
  \begin{itemize}
    \item \emph{Campos obrigatórios:} title.
    \item \emph{Campos opcionais:} author, howpublished, address, month, year, note.
  \end{itemize}

  \begin{lstlisting}
  @booklet{caxton:1993,
    title        = {The title of the work},
    author       = {Peter Caxton},
    howpublished = {How it was published},
    address      = {The address of the publisher},
    month        = {7},
    year         = 1993,
    note         = {An optional note}
  }
  \end{lstlisting}


  \item[@manual] Documentação técnica.
  \begin{itemize}
    \item \emph{Campos obrigatórios:} title.
    \item \emph{Campos opcionais:} author, organization, address, edition, month, year, note, annote.
  \end{itemize}

  \begin{lstlisting}
  @manual{Pinto:1991,
    author  = {Luis Filipe Pinto and Pedro Rangel Henriques},
    title   = {Animador de Especifica\c{c}\~oes {OBLOG}},
    year    = {1991},
    month   = {Sep},
    organization = {gdcc},
    edition = {1.st},
    number  = {UMMAN obl-man},
    annote  = {ambientes desenvolvimento, prog oobjectos, espec formal}
  }
  \end{lstlisting}


  \item[@unpublished] Documento não publicado formalmente.
  \begin{itemize}
    \item \emph{Campos obrigatórios:} author, title, note.
    \item \emph{Campos opcionais:} month, year.
  \end{itemize}

  \begin{lstlisting}
  @unpublished{fudenberg:1988,
    title  = {A theory of learning, experimentation, and equilibrium in games},
    author = {Drew Fudenberg  and David M. Kreps},
    year   = {1988},
    note   = {Unpublished paper}
  }
  \end{lstlisting}
\end{description}


\subsection{Campos de Entrada Bib\TeX}

Nesta subseção explica-se o significado exato de cada campo de um registo bibliográfico obrigatório ou opcional.
A listagem segue a ordem alfabética.

\begin{itemize}
  \item \textbf{address} -  Cidade da editora. No caso de \emph{inproceedings}, se a editora não for informada será o local do evento.
  \item \textbf{annote} - Uma anotação. Não é usado pelos estilos de bibliografia padrão, mas pode ser usado por outros que produzem uma bibliografia anotada.
  \item \textbf{author} - Autor(res) do trabalho. Se existir mais do que um autor devem ser separados por "\emph{and}".
                        Quando um autor apresentar apelido composto devemos colocar o primeiro apelido em primeiro, por exemplo, os autores "Paulo Silva Neto, Joaquim Castro" devem ser cadastrados como "Silva{ }Neto, Paulo and Joaquim Castro";
  \item \textbf{booktitle} - Depende do tipo de entrada. Para \emph{inbook} é o nome do livro, para \emph{incollection} é o nome da colecção, já para \emph{inproceedings} é o nome da publicação da conferência ou evento;
  \item \textbf{chapter} - Número do capítulo;
  \item \textbf{doi} - Digital Object Identifier - representa um sistema de identificação numérico para conteúdo digital, como livros, artigos electrónicos e documentos em geral;
  \item \textbf{edition} - Edição de um livro - por exemplo, "Segundo";
  \item \textbf{editor} -  Nome do(s) editor(es) semelhante ao campo author;
  \item \textbf{howpublished} -  Como o trabalho está disponível;
  \item \textbf{journal} - Nome da revista ou periódico;
  \item \textbf{institution} - A instituição patrocinadora de um relatório técnico.
  \item \textbf{isbn} - Código universal de qualquer livro.
  \item \textbf{issn} - Código universal de qualquer revista.
  \item \textbf{month} - Mês  em que o trabalho foi publicado ou, para um trabalho não publicado, no qual foi escrito;
  \item \textbf{note} -  Observação. Qualquer informação adicional que possa ajudar o leitor;
  \item \textbf{number} -  O número de um periódico, revista, relatório técnico ou de um trabalho numa série.
                        Uma edição de um periódico ou revista é geralmente identificada pelo seu volume e número; a organização que emite um relatório técnico geralmente fornece um número; e às vezes os livros recebem números numa série nomeada;
  \item \textbf{organization} - Organização. No caso de \emph{inproceedings} é o nome do evento;
  \item \textbf{pages} -  Total de páginas do trabalho (mais usado em livros, monografias) ou páginas inicial e final separadas
                          por hífen (mais usado em artigos);
  \item \textbf{publisher} - Editora;
  \item \textbf{school}  - Nome da instituição de ensino;
  \item \textbf{section} -  Secção;
  \item \textbf{series}  - O nome de uma série ou conjunto de livros. Ao citar um livro inteiro, o campo de título fornece o seu título e um campo de série, opcional, fornece o nome de um conjunto de séries ou vários volumes no qual o livro é publicado;
  \item \textbf{subtitle} -  Subtítulo;
  \item \textbf{title} - Título do trabalho;
  \item \textbf{type} - O tipo de um relatório técnico, por exemplo, `` Research Note '';
  \item \textbf{url} - Endereço na Internet;
  \item \textbf{volume} - O volume de um revista ou livro em multi-volume;
  \item \textbf{year} - Ano de publicação ou, para um trabalho não publicado, o ano em que foi escrito. No caso de \emph{inproceedings}, se a editora não for informada será o ano do evento.

\end{itemize}

\newpage
\section{Citação e Inclusão da Bibliografia}
Para se usar uma base de dados bibliográfica e dela se extraírem as referências citadas num documento com recurso à ferramenta
Bib\TeX, é necessário incluir dois comandos especiais no documentos fonte \LaTeX\ conforme se explica a seguir.

\subsection{Citar Referências}

Para citar uma obra bibliográfica específica, descrita por uma entrada específica do ficheiro Bib\TeX,
num determinado ponto  ao longo do texto principal no ficheiro fonte \LaTeX,
é necessário usar o  comando abaixo, onde \texttt{label} se refere à etiqueta específica (a 'label') e única
da entrada Bib\TeX\ respetiva:

\begin{verbatim} \cite{label}\end{verbatim}

Adicionalmente é possível fazer com que uma obra bibliográfica apareça na listagem final das Referências sem ser referenciada
(citada) no texto.
Para esse efeito basta usar no ficheiro fonte \LaTeX\ o comando:
\begin{verbatim} \nocite{label}\end{verbatim}
Usando em vez desse, o comando abaixo:
\begin{verbatim} \nocite{*}\end{verbatim}
consegue-se incluir na lista das Referências Bibliográficas todas as entradas da base de dados Bib\TeX.

\subsection{Inclusão de um ficheiro Bib\TeX}
O comando seguinte  importa o arquivo Bib\TeX  \texttt{'arquivoBib.bib'}  para dele se extrair a lista
de referências a incluir no documento correspondentes às citações bibliográficas.

\begin{verbatim}\bibliography{arquivoBib}\end{verbatim}

É possível no mesmo documento fonte fazer citações de entradas bibliográficas que estejam contidas em bases de dados
Bib\TeX\ diferentes.
Para importar vários arquivos \texttt{.bib}, basta escrever os seus nomes separados por vírgula dentro das chavetas,
a extensão do arquivo não é necessária.
Por exemplo:
\begin{verbatim}\bibliography{minhaBDB,tuaBDB,suaBDB}\end{verbatim}
procura as citações em três bases de dados distintas: \texttt{minhaBDB.bib}, \texttt{tuaBDB.bib} e \texttt{suaBDB.bib}.

\subsection{Estilos Bib\TeX}
O estilo da bibliografia é definido pelo comando exibido abaixo.

\begin{verbatim}\bibliographystyle{ESTILO}\end{verbatim}

O \texttt{ESTILO} define o tipo de saída da Bibliografia que pode ser:
numérico; abreviação do apelido e data; apelido por extenso e data; etc.

As informações exibidas dependem do estilo de bibliografia utilizado, mesmo que a entrada contenha informações sobre a data,
autor, título, editor, o estilo usado pode imprimir apenas o título e o autor.
Este tópico será explicado e exemplificado a seguir.

Como se disse acima, existem vários estilos de Bibliografia em Bib\TeX; os mais utilizados são:

\begin{description}
  \item[\texttt{plain}] -- numérico pela ordem de citação ou ordem alfabética (exemplo de formato: [1]).

  \item[\texttt{alpha}] -- abreviação do apelido dos vários autores e data da publicação (exemplo de formato:[Hen08] ou [AH18]
  ou [AAea16]).

  \item[\texttt{apalike}] --  apelido do primeiro autor e data da publicação (exemplo de formato: (Araujo,18)).
  Note que neste estilo de Bibliografia é necessário utilizar o pacote:
 \begin{verbatim} \usepackage{apalike}\end{verbatim}
\end{description}

\subsubsection{Exemplos dos Estilos de Bibliografia}
Na Figura~\ref{fig:texInput} é exibido um exemplo de um \emph{documento.tex} (neste caso um Artigo em Português
escrito com o encoding \texttt{UTF-8}) que usa várias citações de  Bibliografia descrita nas entradas existentes
no ficheiro \texttt{bibLayout.bib}.\\
Este documento vai servir de exemplo para demonstrar os diferentes \textbf{Estilos} de  Bibliografia em Bib\TeX.

 \begin{figure}[h]
  \centering
  \includegraphics[width=1\textwidth]{img/texInput}
  \caption{Exemplo de um \emph{documento.tex} com Bibliografia}\label{fig:texInput}
\end{figure}

%\subsection*{Estilos de Bibliografia}

  Na Figura~\ref{fig:estiloPlain} podemos ver um exemplo de uma Bibliografia com estilo \emph{Plain}.

  \begin{figure}[h]
  \centering
  \includegraphics[width=0.7\textwidth]{img/estiloPlain}
  \caption{Exemplo de uma Bibliografia com estilo \emph{Plain}}\label{fig:estiloPlain}
  \end{figure}

  Na Figura~\ref{fig:estiloAlpha} podemos ver um exemplo de uma Bibliografia com estilo \emph{Alpha}.

  \begin{figure}[h]
  \centering
  \includegraphics[width=0.7\textwidth]{img/estiloAlpha}
  \caption{Exemplo de uma Bibliografia com estilo \emph{Alpha}}\label{fig:estiloAlpha}
  \end{figure}


  Na Figura~\ref{fig:estiloAlpha} podemos ver um exemplo de uma Bibliografia com estilo \emph{Alpha}.

  \begin{figure}[h]
  \centering
  \includegraphics[width=0.7\textwidth]{img/estiloApalike}
  \caption{Exemplo de uma Bibliografia com estilo \emph{Apalike}}\label{fig:estiloApalike}
  \end{figure}


\end{document}

